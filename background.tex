\chapter{BACKGROUND}
\label{chap:background}
This proposal brings together various technologies synergetically thus
us to reuse various existing tools.
%
By combining a heterogeneous computing framework, virtual private
networking, and IP multicasting, I propose a solution that is based on
well accepted academic and industry standard.
%
In this section, I elaborate on some of these current standards and
their impact on the proposed design.
%
\section{Heterogeneous Computing}
\label{back:heterogeneous}
An increasing number of hardware platforms are incorporating
heterogeneity for both energy and performance improvement.
%
Therefore,  it has been seen that primary host CPU cores are assisted by
various computing units for specialized functions offloading.
%
These specialized accelerators are aimed at speeding up specific
portions of the application code and differ from the general purpose
processor architectures.
%
In this subsection, I explain some of the different types of
heterogeneous platforms available nowadays.\\
%
{\bf Heterogeneous multi-core CPU combinations.}
In this design, high performing CPU cores are integrated with low
performing energy efficient cores.
%
Currently, some mobile and server platforms employ this
architecture~\cite{atom}.
%
The goal is to match the application execution requirements to the
smallest available core that can achieve the execution within required
performance limitations with lowest energy consumption.
%
In other cases, one big processor is utilized for controlling the
scheduler of multiple smaller cores~\cite{nvidia}.
%
Previous works have suggested that this combination can reduce energy
consumption by 39\%~\cite{kumar}.\\
%
{\bf CPU-GPU combinations.}
These platforms aim to accelerate graphics performance by including
specialized hardware to execute graphics functions.
%
The Graphics Processor Units (GPUs) consist of multiple execution units
and local memory that can efficiently process data and task parallel
instruction streams at a very high throughput.
%
This architectural property also make GPUs good candidates for executing
non-graphical functions that can exploit the data and task parallelism
at a large scale leading to evolution of a rich General-Purpose
computing on Graphics Processing Units (GPGPU) based offloading
frameworks.\\
{\bf Hardware accelerators.}
Increasingly specialized hardware accelerators are being integrated on
different platforms to provide faster execution times for
domain-specific application bottleneck functions.
%
The accelerators range from fixed function units to customizable and
programmable hardware (e.g. FPGAs).
%
In conjunction with the processor cores, the specialized hardware
accelerators assist in making the targeted applications perform better
by taking care of the functions that use a majority of core cycles.
%
This model makes accelerator integration a natural choice for platforms
that target specific usages and/or operate under real-time and power
constraints.\\
%
{\bf Remote offloading.}
%One of the contribution in this dissertation is to extend the definition
%of heterogeneity to include a new kind of platform component -- the
%execution units on remote resources.
%
With the advent of mobile era, remote execution to remote resources has
been one of the most common solutions to achieve better application
performance.
%
Due to the compute, memory and power limitations on mobile devices, most
of the applications divide their execution between the portions of local
processing and remote execution handling much of the complex and heavy
task.\\
%
%The proposed architecture aims to extend this model by providing mobile
%applications with the flexibility of using remote resources through a
%well-defined framework, namely OpenCL.
%
In this dissertation, one of the contributions is to extend the scope
of heterogeneity to include a new kind of platform component -- the
compute units on remote devices located in the private or public clouds.
%
Therefore, the proposed multi-layer framework allows for application to
discovery, enumerate, and use the remote devices as if they were local
devices.
%
The framework has the advantage of not only speeding application
execution times and saving mobile device energy consumption but also
allows for the construction of a virtual dynamic platform with tight
integration of heterogeneous components both on and off the host
platform.
%
\section{OpenCL}
\label{back:opencl}
%
Currently, OpenCL (Open Compute Language) is the most well-known
framework for offloading tasks to different compute units within a
heterogeneous host platform.
%
This framework provides an APIs which make it possible for application
developers to dispatch kernels for execution.
%
It is supported by the top chip makers such as Intel, NVIDIA, AMD and
ARM Holdings.
%
At the moment, OpenCL is used in the scientific computing arena as means
to leverage GPUs for high-throughput computations.
%
I implement the proposed design as a subsystem of the OpenCL framework.
%
The extension makes it possible for the OpenCL framework to discover
accelerators located on remote resources across the network.
%
Once these remote OpenCL devices are discovered, the application
developers is able to use the same interface to transparently offload a
kernel as if it was a part of the local platform.
%
\section{Secure Offloading with P2PVPN}
\label{back:p2pvpn}
Previous works have not focused on dealing with the privacy implications
of offloading mobile computation to the cloud for remote execution.
%
One solution can be to use socket layer encryption such as SSL/TLS when
sending data to the cloud, but the encryption overhead has not been
studied in past research.
%
As privacy concerns in mobile devices grow, this dissertation
recognizes that a robust mechanism to ensure privacy is a fundamental
requirement.
%
To address the privacy issue, the proposed design integrates a social
peer-to-peer virtual private network (SocialVPN~\cite{socialvpn}).
%
SocialVPN automatically discovers social peers on the Internet and
creates a private network consisting solely of these trusted peers.
%
It also handles the cumbersome task of cryptographic key management in a
distributed fashion thus enabling a robust, secure communication layer
with no single point of failure.
%
By emulating a virtual LAN, peers have private IP addresses to each
other allowing them to freely communicate even if they are both behind
network address translators (NAT) and certain firewalls.
%
Hung et al.~\cite{hung} suggested using a centralized VPN to ensure data
privacy.
%
Centralized VPNs (e.g. OpenVPN, Cisco VPN, L2TP) typically function by
allowing the VPN client to connect to a VPN gateway, which forwards UP
requests to the internal private network on behalf of the clients.
%
This model does not guarantee any privacy between the gateway and the
destination machine because it does not provide end-to-end encryption.
%
This is exacerbated in a public cloud where you have different and
unknown organizations sharing the same private network.
%
This lack of IP level data privacy is a primary reason that many
organizations are hesitant to run their private workloads in the public
clouds\cite{brodkin}.
%
SocialVPN addresses this issue by enforcing encrypted end-to-end
communication.\\
%
The other important aspect of SocialVPN is that it gives the user
complete control over their virtual private network.
%
A user can create a virtual network consisting only of his/her own
personal computing devices.
%
For example, Alice can create a VPN consisting of only 4 nodes: her
mobile phone, her laptop, her workstation at the office, and a VM
instance running on Amazon EC2.
%
Alice can also expand the VPN to include her family members, and closest
friends or scale it down dynamically.
%
If Alice only trusts her own personal resources, she simply limits her
VPN to allow connectivity to her machines, and the framework will only
have private UP connectivity to her nodes thereby making them the only
candidates for remote offloading.
%
If she wants to access to more resources, she can allow private IP
access to nodes belonging to her trusted friends and family, hence
enabling the framework to utilize them  as compute endpoints.
%
The assumption here is that all of these nodes will be running SocialVPN
software stack along with an OpenCL RPC-service to support the remote
execution.
%
\section{IP Multicasting in Private Networks}
\label{back:ipmulticast}
The Internet protocol does support multicasting; however, routers
usually block multicast traffic; hence multicasting is widely used for
decentralized service discovery within private networks.
%
For example, most network printers support local network discovery
through well-known standards such as Bonjour~\cite{bonjour}, or
Universal Plug and Play (UPnP).
%
Hence, all modern operating systems as well as smart appliances (i.e.
TVs, NAS boxes) rely on these IP multicast-based services mechanisms to
discover and communicate with each other.
%
Through integration with SocialVPN, the proposed offloading framework is
able to reues these IP multicasting techniques as the foundation for the
service discovery system and allow it to span across wide area networks.
%
With this decentralized approach, it is no longer necessary to register
services in a centralized directory.
%
When the mobile device decides to offload some computation, it sends an
IP multicast query over the network, SocialVPN automatically handles
this packet and sends it to all trusted peers in the private network.
%
Eligible peers are therefore able to advertise themselves to the
requestor.
%
This enables dynamic service discovery without having to hardcoded
endpoints in configuration files or depend on a directory service.
%
IP multicast support in SocialVPN enables us to follow the same
decentralized services discovery techniques that are widely used in the
private networks for network printers, DNLA compatible devices (e.g. TVs)
and storage boxes.
%


