\begin{abstract}
%Mobile computing is becoming the preferred method of personal computing
%environment for millions of users.
%
%Furthermore, the growth of the mobile device market has been inspired by the
%hundreds of thousands of mobile applications such as social networking,
%location-based services, image processing, augmented reality, and
%face/speech recognition.
%
%In order to meet the increasing demands of these computationally-intense
%applications, recent mobile platforms have been augmented with multi-core CPUs,
%more powerful GPUs and other special types of hardware accelerators.
%
%Despite of these enhancements to mobile platform's hardware,
%however, the inescapable fact is that their limited capacities of the
%batteries will always serve as a bottleneck while hindering the mobile
%platforms from utilizing their computing capabilities.\\
%
%To address this restriction, there have been research efforts on remote
%offloading systems which seek intelligent ways to enable mobile
%platform developers to leverage computing capabilities of more powerful
%resources over the network. 
%
%Even though existing approaches provide core mechanisms to transform
%typical mobile applications to offloading-enabled applications through
%various granularities of partitioning and/or migration, they still lack
%considerations for service discovery mechanisms while assuming the
%availability of remote computing nodes with static endpoints.
%
%Moreover, they have not investigated data privacy and secure
%communication between the mobile client and remote resources, which can
%be a crucial flaw for mobile computing environments.\\
%
%This dissertation presents a novel framework which enables remote workload
%offloading to external resources within a mobile user's Social Device
%Network in which trusted remote computing resources such as family
%or friend's desktops or laptops are aggregated regardless of the user's
%mobility.
%
%The proposed system accomplishes this by 1) utilizing a peer-to-peer
%virtual private networking technique as a substrate for the discovery
%and configuration of trusted remote resources, and 2) extending OpenCL
%framework, which is an open standard of parallel programming for
%heterogeneous computing environments, to support remote offloading using
%the TCP/IP networking stack. 
%
%This work thus broadens the range of heterogeneous
%computing to remote computing capabilities in the network, where
%offloading servers and services are dynamically discovered.\\
%
%The prototype implementation of the proposed framework is evaluated,
%with regard to end-to-end application performance and energy
%consumption in mobile devices, through a variety of network
%configurations representing local and wide area network, and various
%levels of remote computing capabilities such as typical CPUs, GPUs as
%well as Amazon EC2 instances.
%
%According to the evaluation, the proposed architecture achieves more
%energy efficient performance by offloading than executing locally
%depending on the characteristics of mobile workloads and network
%conditions.\\
%
%Based on the evaluation results, mobile workloads are characterized for
%the suitability of offloading from the perspective of computation to
%communication ratio which is a comprehensive measurement mirroring
%network conditions and workload characteristics.
%
%In addition, this dissertation proposes applying machine learning
%techniques to runtime schedulers for mobile offloading framework.
%
%By adopting machine learning techniques to remote offloading
%scheduling problems, a scheduler can be automatically trained from
%previous offloading performance and make decisions on whether mobile
%workloads should be offloaded or executed locally according to past
%behaviors and current conditions.
%
%While running various machine learning algorithms, the evaluation shows the
%feasibility of adopting machine learning techniques into scheduling
%problems for mobile offloading framework.\\
%
%As the future work, the current resource discovery technique will be 
%further extended into the consideration of keeping track of a more 
%complex set of conditions of multiple remote computing resources such 
%as network latency, bandwidth, and computing capabilities of remote 
%resources, and the provision of the most appropriate resource in 
%accordance with network conditions and mobile application requirements.
%
%Also, the machine learning-based runtime scheduler will be modularized
%so that it provides well-defined APIs, and the proposed runtime scheduler
%can be plugged and played for various types of adaptive scheduling
%problems.
%
%In doing so, it is possible to characterize the performance, benefits,
%and overhead of different types of machine learning algorithms in online
%schedulers for mobile offloading frameworks.
%
%As part of the modularization of the machine learning-based runtime
%scheduler, I am currently working on the online scheduler for Java-based
%on-demand code offloading system.
%
Mobile computing is becoming the preferred method of personal computing
for millions of users.
%
In order to meet the increasing demands of computationally-intensive
applications, recent mobile platforms have been augmented with
multi-core CPUs, powerful GPUs, and special types of hardware
accelerators.
%
Despite these enhancements to the hardware of mobile platforms,
their limited battery capacities of the batteries and small form factor
remain a bottleneck, hindering mobile platforms from utilizing their
computing capabilities.\\
%
To address this restriction, there have been research efforts on remote
offloading systems which seek intelligent ways to enable mobile
platforms to leverage computing capabilities of more powerful resources
over the network.
%
Even though existing approaches provide core mechanisms to transform
typical mobile applications to offloading-enabled applications, they
still lack service discovery mechanisms while assuming the availability
of remote computing nodes with static endpoints.
%
Moreover, data privacy and secure communication between the mobile
client and remote resources are of increasing importance for secure
computing on mobile environments.\\
%
This dissertation presents a novel framework which enables remote
workload offloading to external resources within a social virtual private
network defined by the mobile user, in which trusted remote computing resources
are aggregated in a virtual network regardless of user mobility.
%
The proposed system accomplishes this by utilizing a peer-to-peer
virtual private networking technique as a substrate for the discovery,
configuration of trusted remote resources, and secure communication
between the mobile device and remote resources.
% 
%and extending OpenCL
%framework, which is an open standard of parallel programming for
%heterogeneous computing environments, to support remote offloading using
%the TCP/IP networking stack.
%
Based upon the evaluation on the performance of the proposed offloading
framework, various mobile workloads are characterized for the
suitability of offloading from the perspective of computation to
communication ratio which is a comprehensive measurement mirroring
network conditions and workload characteristics.\\
%
In addition, this dissertation proposes applying machine learning
techniques to runtime schedulers for mobile offloading frameworks.
%
By adopting machine learning techniques to remote offloading scheduling
problems, decisions on offloading do not rely on application-dependent
parameters or predefined static scheduling policies. 
%
Instead, the scheduler can automatically learn the offloading effectiveness from
previous offloading experiences and dynamically make decisions on
whether mobile workloads should be offloaded or executed locally
according to the current conditions at runtime.
%
\end{abstract}
