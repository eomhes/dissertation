% The Editorial Office Requirements for the Table of Contents cause a significant problem 
%in Latex if there is only one Appendix. The Appendix is no longer labeled "A" in the TOC
%but has the word "APPENDIX" placed in front of the title of the Appendix. This can be done
%without issue IF nothing needs to be numbered by LaTeX in the Appendix. Unfortunately, most of the time
%something needs to be numbered in that single Appendix. For this reason we have included the IFTHENELSE switch
%found in this document and at the beginning of AppendixA. We assume that if you have any appendices, that you have more than one.
%So the default setting is noa = 2 (number of appendices = 2). Note: you don't need the actual number of appendices here
%1 or 2 are the only relevant numbers. You just make sure to input the Appendices you do have in this file.
%
%If, however, you DO only have one appendix change the line:
%
%\setcounter{noa}{2} to
%
%\setcounter{noa}{1}
%
%And comment (or delete) all of the input{AppendixB} commands except the first one.
%Then open the AppendixA.tex file and continue there.

%you can add/substract individual appendices through by using the /include{appendix'X'}
% and creating/deleting the appropriate files
\appendix %
\clearpage%
\newcounter{noa} % noa= no. of appendices ... set to 1 for 1 and more otherwise.
\setcounter{noa}{2} % ........................... CHANGE VALUE ONLY HERE
\ifthenelse{\value{noa} = 1}
%...................then
{}
%...................else
{\addtocontents{toc}{\protect\addvspace{10pt}\protect\noindent \protect APPENDIX}}
%...................
%If you have a single appendix, you need to change {\chapter*{APPENDIX: THIS IS THE FIRST APPENDIX}
%to {\chapter*{APPENDIX: YOUR APPENDIX TITLE HERE} if you have two or more appendices
%you need to change {\chapter{THIS IS THE FIRST APPENDIX}} to
%{\chapter{YOUR APPENDIX TITLE HERE}}
%
%If you make these changes correctly Latex will complain bitterly about the additions to the TOC
%but will make them correctly in a manner acceptable to the Editorial Office.

\ifthenelse{\value{noa} = 1}
%...................then
{\chapter*{APPENDIX: THIS IS THE FIRST APPENDIX}
\addcontentsline{toc}{chapter}{APPENDIX: THIS IS THE FIRST APPENDIX}
\chaptermark{Appendix}
\markboth{Appendix}{Appendix}
\setcounter{chapter}{1}}
%...................else
{\chapter{THIS IS THE FIRST APPENDIX}}
%...................



Lorem ipsum dolor sit amet, consectetuer adipiscing elit. Maecenas
eget magna. Aenean et lorem. Ut dignissim neque at nisi. In hac
habitasse platea dictumst. In porta ornare eros. Nunc eu ante. In
non est vehicula tellus cursus suscipit. Proin sed libero. Sed risus
enim, eleifend in, pellentesque ac, nonummy quis, nulla. Phasellus
imperdiet libero nec massa. Ut sapien libero, adipiscing eu,
volutpat porttitor, ultricies eget, nisi. Sed odio. Suspendisse
potenti. Duis dolor augue, viverra id, porta in, dignissim id, nisl.
Vivamus blandit cursus eros. Maecenas sit amet urna sit amet orci
nonummy pharetra.

Praesent cursus nibh et mauris. In aliquam felis sit amet ligula.
Nulla faucibus nisl eget nisl. Aliquam tincidunt. Mauris eget elit
sed massa luctus posuere. Pellentesque suscipit. In odio urna,
semper ut, convallis ut, porta et, nibh. Nulla sodales metus nec
velit posuere gravida. Cras tristique. Etiam urna risus, accumsan
ut, placerat sed, iaculis id, est.

Nullam mi. Pellentesque habitant morbi tristique senectus et netus
et malesuada fames ac turpis egestas. Duis vitae metus in massa
hendrerit rhoncus. Fusce tortor justo, laoreet eu, facilisis at,
gravida et, felis. Donec imperdiet mollis erat. Integer tempus nulla
ac lorem. Fusce porttitor. Aenean quis arcu. Morbi consectetuer, leo
eu mollis elementum, urna massa malesuada risus, euismod tempor
lorem elit ut mauris. Cras elit orci, facilisis ac, mattis iaculis,
cursus ac, augue. Donec eget nisl. Pellentesque fermentum sodales
nibh. Vivamus non risus. Donec est libero, tincidunt sit amet,
pretium vitae, blandit sed, tellus. Nunc diam risus, interdum sed,
laoreet quis, varius ac, turpis. In et purus eget nibh vehicula
rhoncus. Aenean et neque. Praesent nisl nisi, tempus quis, nonummy
ac, auctor a, neque. Suspendisse et metus. Suspendisse non metus eu
mauris auctor sagittis.
 %
\chapter{AN EXAMPLE OF A HALF TITLE PAGE}%
\label{appendixB}

\clearpage %remove this command if your appendix doesn't start with a landscaped page!!!!!
\thispagestyle{plain}
\begin{landscape}
\begin{figure}

  \begin{center}
    \includegraphics[width=6in]{LaTeX2e_logo.eps}
    \caption{\LaTeX 2\ensuremath{\epsilon.} logo}\label{biglogo}
  \end{center}
\end{figure}
\end{landscape}


This is how a section should look if the first page is a landscape page.
Lorem ipsum dolor sit amet, consectetuer adipiscing elit. Ut sit
amet nulla. Integer mauris turpis, dapibus ac, auctor non, vehicula
sit amet, magna. Suspendisse eu tellus. Etiam porta. Donec magna.
Donec ut dui. In hac habitasse platea dictumst. Nullam suscipit, mi
at adipiscing commodo, lorem erat scelerisque erat, non pulvinar leo
mi eu metus. Phasellus id felis. Sed quam purus, molestie quis,
ultrices nec, dictum at, magna. Proin viverra viverra ante.

Maecenas sagittis magna quis ligula. Duis vestibulum mi a felis.
Aenean accumsan mattis massa. Nullam lacus sem, consectetuer non,
condimentum sit amet, pharetra ac, odio. Morbi nisi magna, tincidunt
sed, placerat nec, tincidunt id, lectus. Donec ac dui non mauris
vulputate aliquam. Nullam scelerisque congue pede. Integer ipsum.
Vestibulum auctor. Suspendisse eget leo id libero cursus dictum. Sed
malesuada. Aliquam imperdiet. Donec dui metus, porta eu, aliquet
vel, vulputate vitae, lacus.

Nulla quis purus id turpis luctus feugiat. Fusce feugiat. Proin
felis. Morbi elit est, fermentum in, tincidunt vitae, convallis vel,
orci. Vestibulum justo. Suspendisse non nisl. Pellentesque pretium
adipiscing elit. Phasellus fermentum consequat augue. Sed pede nisl,
fermentum vel, vulputate id, sollicitudin sed, ligula. Cras
suscipit, quam et euismod sagittis, nisl felis gravida felis, quis
pulvinar purus est vel pede. Suspendisse mattis est ac nunc.
Curabitur rutrum, turpis sit amet commodo tempus, metus lorem
commodo lectus, eget fringilla justo nisi et purus. Ut quam sapien,
vehicula quis, rhoncus non, sagittis nec, risus.

Donec eget augue ac lacus adipiscing porta. Maecenas pede. Vivamus
molestie. Duis condimentum ligula auctor pede. Nullam ullamcorper
rhoncus erat. Ut ornare interdum urna. Suspendisse potenti.
Curabitur mattis mauris nec risus. Aenean iaculis turpis eu tortor.
Donec nec ante non mauris pellentesque fringilla.

Phasellus vitae dui id orci sodales cursus. Curabitur sed nulla quis
mauris tincidunt iaculis. Vivamus semper semper orci. Phasellus
suscipit ante vitae leo. Sed arcu ipsum, condimentum id, luctus in,
sodales eu, magna. In dictum, arcu quis pharetra vestibulum, ante
enim placerat lacus, vitae placerat est leo vitae elit. Pellentesque
bibendum enim vulputate eros. Nunc laoreet. Pellentesque habitant
morbi tristique senectus et netus et malesuada fames ac turpis
egestas. Praesent purus odio, euismod sit amet, aliquam a, volutpat
in, augue. Phasellus id massa. Suspendisse suscipit ligula pharetra
dolor. Pellentesque vel pede.

Aliquam pharetra est sit amet magna. Aliquam varius. Donec eu lectus
et nisl iaculis porttitor. Morbi mattis, mauris sed luctus
hendrerit, nulla velit molestie dolor, ac volutpat urna augue vel
quam. Maecenas pellentesque libero et massa. Integer vestibulum,
lacus at mattis euismod, nisl arcu commodo lectus, ut euismod dolor
ligula sit amet libero. Nam in ligula sit amet ante eleifend
aliquet. Phasellus feugiat erat at nulla. Proin in lectus. Proin
laoreet leo laoreet leo congue lacinia. Quisque non diam sit amet
enim ultrices commodo. Praesent fermentum lectus sed ligula. Integer
pulvinar accumsan pede. Quisque molestie ligula eget odio.
Vestibulum ante ipsum primis in faucibus orci luctus et ultrices
posuere cubilia Curae;
 % UNCOMMENT APPENDICES IF THERE ARE MORE APPENDICES
\chapter{DERIVATION OF THE $\Upsilon$ FUNCTION}%
\label{appendixC}

%\clearpage %remove this command if your appendix doesn't start with a landscaped page!!!!!
%\thispagestyle{plain}
%\begin{landscape}
%\begin{figure}

 % \begin{center}
  %  \includegraphics[width=6in]{LaTeX2e_logo.eps}
   % \caption{\LaTeX 2\ensuremath{\epsilon.} logo}\label{biglogo}
  %\end{center}
%\end{figure}
%\end{landscape}

%%%%%%%%%%%%%%%%%%%%%%%%%%%%%%%%%%%%%%%%%%%%%%%%%%%%%%%%%%%%%%%%%%%%%%%%%%%%%%%%%%%%%%%%%%%%%%%%%%


%ADD LABEL

%%%%%%%%%%%%%%%%%%%%%%%%%%%%%%%%%%%%%%%%%%%%%%%%%%%%%%%%%%%%%%%%%%%%%%%%%%%%%%%%%%%%%%%%%%%%%%%%%%

\proposition{The Upsilon Function}\label{first}

(1) If $\beta>0$ and $\alpha\neq0$, then for all $n\geq-1$,

$$I_{n}(c;\alpha; \beta; \delta) = - \frac{e^{\alpha c}}{\alpha} \sum_{i=0}^{n}(\frac{\beta}{\alpha})^{n-i} Hh_{i}(\beta c -\delta)$$
$$+ (\frac{\beta}{\alpha})^{n+1} \frac{\sqrt{2 \pi}}{\beta} e^{\frac{\alpha \delta}{\beta}+\frac{\alpha^{2}}{2\beta^{2}}} \phi(-\beta c + \delta + \frac{\alpha}{\beta})$$

(2) If $\beta<0$ and $\alpha<0$, then for all $x \geq -1$

$$I_{n}(c;\alpha; \beta; \delta) = - \frac{e^{\alpha c}}{\alpha} \sum_{i=0}^{n}(\frac{\beta}{\alpha})^{n-i} Hh_{i}(\beta c -\delta)$$
$$- (\frac{\beta}{\alpha})^{n+1} \frac{\sqrt{2 \pi}}{\beta} e^{\frac{\alpha \delta}{\beta}+\frac{\alpha^{2}}{2\beta^{2}}} \phi(\beta c - \delta - \frac{\alpha}{\beta})$$

\begin{proof}{Case 1.}
$\beta>0$ and $\alpha\neq0$. Since, for any constant $\alpha$ and $n \geq 0$, $e^{\alpha x} Hh_{n}(\beta x - \delta) \rightarrow 0$ as $x \rightarrow \infty$ thanks to (B4), integration by parts leads to

$$I_{n}=-\frac{1}{\alpha}Hh(\beta c -\delta) e^{\alpha c} + \frac{\beta}{\alpha}\int_{c}^{\infty} e^{\alpha x} Hh_{n-1}(\beta c - \delta)dx$$

In other words, we have a recursion, for $n \geq 0$, $I_{n}=-(e^{\alpha c}{\alpha})Hh_{n}(\beta c - \delta) + (\frac{\beta}{\alpha})I_{n-1}$ with

$$I_{-1}=\sqrt{2 \pi} \int_{c}{\infty}e^{\alpha x}\varphi(-\beta x +\delta)dx$$

$$=\frac{\sqrt{2 \pi}}{\beta} e^{\frac{\alpha \delta}{\beta}+\frac{\alpha^{2}}{2 \beta^{2}}}\phi(-\beta c + \delta +\frac{\alpha}{\beta})$$

Solving it yields, for $n \geq -1$,

$$I_{n}=-\frac{e^{\alpha c}}{\alpha}\sum_{i=0}^{n}(\frac{\beta}{\alpha})^{i}Hh_{n-i}(\beta c+\delta) + (\frac{\beta}{\alpha})^{n+1}I_{-1}$$

$$=-\frac{e^{\alpha c}}{\alpha}\sum_{i=0}^{n}(\frac{\beta}{\alpha})^{n-i} Hh_{i}(\beta c+\delta)$$
$$+ (\frac{\beta}{\alpha})^{n+1}\frac{\sqrt{2 \pi}}{\beta} e^{\frac{\alpha \delta}{\beta}+\frac{\alpha^{2}}{2 \beta^{2}}}\phi(-\beta c + \delta +\frac{\alpha}{\beta})$$

where the sum over an empty set is defined to be zero.
\end{proof}

Case2. $\beta<0$ and $\alpha<0$. In this case, we must also have, for $n \geq 0$ and any constant $\alpha<0, e^{\alpha x}Hh_{n}(\beta x -\delta) \rightarrow 0$ as
$x \rightarrow \infty$, thanks to (B5). Using integration by parts, we again have the same recursion, for $n \geq 0, I_{n}=-(e^{\alpha c}/\alpha)Hh_{n}(\beta c - \delta)+(\beta / \alpha)I_{n-1}$, but with a different initial condition

$$I_{-1}=\sqrt{2 \pi}\int_{c}^{\infty}e^{\alpha x}\varphi(-\beta x + \delta)dx$$

$$=-\frac{\sqrt{2 \pi}}{\beta} exp\{\frac{\alpha \delta}{\beta}+\frac{\alpha^{2}}{2 \beta^{2}}\}\phi(\beta c - \delta -\frac{\alpha}{\beta})$$

Solving it yields (B8), for $n \geq -1$.

Finally, we sum the double exponential and the normal random variables

Proposition B.3.

Suppose $\{\xi_{1},\xi_{2},...\}$ is a sequence of i.i.d. exponential random variables with rate $\eta>0$, and Z is a normal variable with distribution $N(0,\sigma^{2})$. Then for every $ n \geq 1$, we have: (1) The density functions are given by:

$$f_{Z+\sum_{i=1}^{n}\xi_{i}}(t)=(\sigma\eta)^{n}\frac{e^{(\sigma\eta)^{2}/2}}{\sigma\sqrt{2\pi}}e^{-t\eta}Hh_{n-1}(-\frac{t}{\sigma}+\sigma\eta)$$

$$f_{Z-\sum_{i=1}^{n}\xi_{i}}(t)=(\sigma\eta)^{n}\frac{e^{(\sigma\eta)^{2}/2}}{\sigma\sqrt{2\pi}}e^{-t\eta}Hh_{n-1}(\frac{t}{\sigma}+\sigma\eta)$$

(2) The tail probabilities are given by

$$P(Z+\sum_{i=1}^{n}\xi_{i}\geq x) = (\sigma\eta)^{n}\frac{e^{(\sigma\eta)^{2}/2}}{\sigma\sqrt{2\pi}}e^{-t\eta}I_{n-1}(x;-\eta,-\frac{1}{\sigma},-\sigma\eta)$$

$$P(Z-\sum_{i=1}^{n}\xi_{i}\geq x) = (\sigma\eta)^{n}\frac{e^{(\sigma\eta)^{2}/2}}{\sigma\sqrt{2\pi}}e^{-t\eta}I_{n-1}(x;\eta,\frac{1}{\sigma},-\sigma\eta)$$

Proof. Case 1. The densities of $Z+\sum_{i=1}^{n}\xi_{i}$, and $Z-\sum_{i=1}^{n}\xi_{i}$. We have

$$f_{Z+\sum_{i=1}^{n}\xi_{i}}(t)=\int_{-\infty}^{\infty}f_{\sum_{i=1}^{n}\xi_{i}}(t-x)f_{Z}(x)dx$$

$$=e^{-t\eta}(\eta^{n})\int_{-\infty}{t}\frac{e^{x\eta}(t-x)^{n-1}}{(n-1)!}\frac{1}{\sigma\sqrt{2\pi}}e^{-x^{2}/(2\sigma^{2})}dx$$

$$=e^{-t\eta}(\eta^{n})e^{(\sigma\eta)^{2}/(2)}\int_{-\infty}{t}\frac{(t-x)^{n-1}}{(n-1)!}\frac{1}{\sigma\sqrt{2\pi}}e^{-(x-\sigma^{2}\eta)^{2}/(2\sigma^{2})}dx$$

Letting $y=(x-\sigma^{2}\eta)/\sigma$ yields

$$f_{Z+\sum_{i=1}^{n}\xi_{i}}(t)=e^{-t\eta}(\eta^{n})e^{(\sigma\eta)^{2}/(2)}\sigma^{n-1}$$

$$\times\int_{-\infty}^{t/\sigma-\sigma\eta}\frac{(t/\sigma - y -\sigma\eta)^{n-1}}{(n-1)!}\frac{1}{\sqrt{2\pi}}e^{-y^{2}/2}dy$$

$$=\frac{e^{(\sigma\eta)^{2}/2}}{\sqrt{2\pi}}(\sigma^{n-1}\eta^{n})e^{-t\eta}Hh_{n-1}(-t/\sigma + \sigma\eta)$$

because $(1/(n-1)!)\int_{-\infty}{a}(a-y)^{n-1}e^{-y^{2}/2}dy=Hh_{n-1}(a)$. The derivation of $f_{Z+\sum_{i=1}^{n}\xi_{i}}(t)$ is similar.

Case 2. $P(Z+\sum_{i=1}^{n}\xi_{i}\geq x)$ and $P(Z-\sum_{i=1}^{n}\xi_{i}\geq x)$. From (B9), it is clear that

$$P(Z+\sum_{i=1}^{n}\xi_{i}\geq x)=\frac{(\sigma\eta)^{n}e^{(\sigma\eta)^{2}/2}}{\sigma\sqrt{2\pi}}\int_{x}^{\infty}e^{(-i\eta)}Hh_{n-1}(-\frac{t}{\sigma}+\sigma\eta)dt$$

$$=\frac{(\sigma\eta)^{n}e^{(\sigma\eta)^{2}/2}}{\sigma\sqrt{2\pi}}I_{n-1}(x;-\eta,-\frac{1}{\sigma},-\sigma\eta)dt$$

by (B6). We can compute
$P(Z-\sum_{i=1}^{n}\xi_{i}\geq x)$ similarly.

\theorem{Theorem} With $\pi_{n}:= P(N(t)=n)=e^{-\lambda T}(\lambda T)^{n}/n!$ and $I_{n}$ in Proposition \ref{first}.
, we have

$$P(Z(T)\geq a)=\frac{e^{(\sigma \eta_{1})^{2} T/2}}{\sigma \sqrt{2 \pi T}} \sum_{n=1}^{\infty} \pi_{n} \sum_{k=1}^{n} P_{n,k}(\sigma\sqrt{T}\eta_{1})^{k}\times I_{k-1}(a-\mu T; -\eta_{1},-\frac{1}{\sigma\sqrt{T}},-\sigma\eta_{1}\sqrt{T})$$

$$+\frac{e^{(\sigma\eta_{2})^{2}T/2}}{\sigma\sqrt{2\pi T}}\sum_{n=1}^{\infty}\pi_{n}\sum_{k=1}^{n}Q_{n,k}(\sigma\sqrt{T}\eta_{2})^{k}$$

$$\times I_{k-1}(a-\mu T; \eta_{2},\frac{1}{\sigma\sqrt{T}},-\sigma\eta_{2}\sqrt{T})$$

$$+\pi_{0}\phi(-\frac{a-\mu T}{\sigma\sqrt{T}})$$

Proof by the decomposition (B2)

$$P(Z(T) \geq a)= \sum_{n=0}^{\infty}\pi_{n} P(\mu T +\sigma\sqrt{T} Z + \sum_{j=1}^{n}Y_{j} \geq a)$$

$$=\pi_{0}P(\mu T +\sigma\sqrt{T} Z  \geq a)$$

$$+\sum_{n=1}^{\infty}\pi_{n}\sum_{k=1}^{n}P_{n,k} P(\mu T +\sigma\sqrt{T} Z + \sum_{j=1}^{n}\xi_{j}^{+} \geq a)$$

$$+\sum_{n=1}^{\infty}\pi_{n}\sum_{k=1}^{n}Q_{n,k} P(\mu T +\sigma\sqrt{T} Z - \sum_{j=1}^{n}\xi_{j}^{-} \geq a)$$

The result now follows via (B11) and (B12) for $\eta_{1} > 1$ and $\eta_{2} >0$.


 %
%\documentclass{article}
%\usepackage[all]{xy}
%\usepackage{amssymb}
%\usepackage{amsmath}
%\usepackage{amsfonts}
%\usepackage{amsthm}
%\usepackage{amscd}
%\usepackage{eucal}
%\usepackage[dvips]{epsfig}
%\usepackage{graphicx}
%\usepackage{ulem}
%\usepackage{wrapfig}
%\addtolength{\hoffset}{-2cm}
%\addtolength{\topmargin}{-2.8cm}
%\addtolength{\textwidth}{3 cm}
%\addtolength{\textheight}{6.2 cm}
%
%\def\ii{{\bf i}}
%\def\jj{{\bf j}}
%\def\kk{{\bf k}}
%\def\aa{{\bf a}}
%\def\bb{{\bf b}}
%\def\nn{{\bf n}}
%\def\uu{{\bf u}}
%\def\vv{{\bf v}}
%\def\rr{{\bf r}}
%\def\ff{{\bf F}}
%
%\begin{document}

%%%%%%%%%%%%%%%%%%%%%%%%%%%%%%%%%%%%%%%%%%%%%%%

\chapter{DERIVATION OF THE $\Upsilon$ FUNCTION}%
\label{appendixB}

%%\clearpage %remove this command if your appendix doesn't start with a landscaped page!!!!!
%%\thispagestyle{plain}
%%\begin{landscape}
%%\begin{figure}

 %% \begin{center}
  %%  \includegraphics[width=6in]{LaTeX2e_logo.eps}
   %% \caption{\LaTeX 2\ensuremath{\epsilon.} logo}\label{biglogo}
  %%\end{center}
%%\end{figure}
%%\end{landscape}

%%%%%%%%%%%%%%%%%%%%%%%%%%%%%%%%%%%%%%%%%%%%%%%%%%%%%%%%%%%%%%%%%%%%%%%%%%%%%%%%%%%%%%%%%%%%%%%%%%


%ADD LABEL


Proposition B.3.

Suppose $\{\xi_{1},\xi_{2},...\}$ is a sequence of i.i.d. exponential random variables with rate $\eta>0$, and Z is a normal variable with distribution $N(0,\sigma^{2})$. Then for every $ n \geq 1$, we have: (1) The density functions are given by:

$$f_{Z+\sum_{i=1}^{n}\xi_{i}}(t)=(\sigma\eta)^{n}\frac{e^{(\sigma\eta)^{2}/2}}{\sigma\sqrt{2\pi}}e^{-t\eta}Hh_{n-1}(-\frac{t}{\sigma}+\sigma\eta)$$

$$f_{Z-\sum_{i=1}^{n}\xi_{i}}(t)=(\sigma\eta)^{n}\frac{e^{(\sigma\eta)^{2}/2}}{\sigma\sqrt{2\pi}}e^{-t\eta}Hh_{n-1}(\frac{t}{\sigma}+\sigma\eta)$$

(2) The tail probabilities are given by

$$P(Z+\sum_{i=1}^{n}\xi_{i}\geq x) = (\sigma\eta)^{n}\frac{e^{(\sigma\eta)^{2}/2}}{\sigma\sqrt{2\pi}}e^{-t\eta}I_{n-1}(x;-\eta,-\frac{1}{\sigma},-\sigma\eta)$$

$$P(Z-\sum_{i=1}^{n}\xi_{i}\geq x) = (\sigma\eta)^{n}\frac{e^{(\sigma\eta)^{2}/2}}{\sigma\sqrt{2\pi}}e^{-t\eta}I_{n-1}(x;\eta,\frac{1}{\sigma},-\sigma\eta)$$

Proof. Case 1. The densities of $Z+\sum_{i=1}^{n}\xi_{i}$, and $Z-\sum_{i=1}^{n}\xi_{i}$. We have

$$f_{Z+\sum_{i=1}^{n}\xi_{i}}(t)=\int_{-\infty}^{\infty}f_{\sum_{i=1}^{n}\xi_{i}}(t-x)f_{Z}(x)dx$$

$$=e^{-t\eta}(\eta^{n})\int_{-\infty}{t}\frac{e^{x\eta}(t-x)^{n-1}}{(n-1)!}\frac{1}{\sigma\sqrt{2\pi}}e^{-x^{2}/(2\sigma^{2})}dx$$

$$=e^{-t\eta}(\eta^{n})e^{(\sigma\eta)^{2}/(2)}\int_{-\infty}{t}\frac{(t-x)^{n-1}}{(n-1)!}\frac{1}{\sigma\sqrt{2\pi}}e^{-(x-\sigma^{2}\eta)^{2}/(2\sigma^{2})}dx$$

Letting $y=(x-\sigma^{2}\eta)/\sigma$ yields

$$f_{Z+\sum_{i=1}^{n}\xi_{i}}(t)=e^{-t\eta}(\eta^{n})e^{(\sigma\eta)^{2}/(2)}\sigma^{n-1}$$

$$\times\int_{-\infty}^{t/\sigma-\sigma\eta}\frac{(t/\sigma - y -\sigma\eta)^{n-1}}{(n-1)!}\frac{1}{\sqrt{2\pi}}e^{-y^{2}/2}dy$$

$$=\frac{e^{(\sigma\eta)^{2}/2}}{\sqrt{2\pi}}(\sigma^{n-1}\eta^{n})e^{-t\eta}Hh_{n-1}(-t/\sigma + \sigma\eta)$$

because $(1/(n-1)!)\int_{-\infty}{a}(a-y)^{n-1}e^{-y^{2}/2}dy=Hh_{n-1}(a)$. The derivation of $f_{Z+\sum_{i=1}^{n}\xi_{i}}(t)$ is similar.

Case 2. $P(Z+\sum_{i=1}^{n}\xi_{i}\geq x)$ and $P(Z-\sum_{i=1}^{n}\xi_{i}\geq x)$. From (B9), it is clear that

$$P(Z+\sum_{i=1}^{n}\xi_{i}\geq x)=\frac{(\sigma\eta)^{n}e^{(\sigma\eta)^{2}/2}}{\sigma\sqrt{2\pi}}\int_{x}^{\infty}e^{(-i\eta)}Hh_{n-1}(-\frac{t}{\sigma}+\sigma\eta)dt$$

$$=\frac{(\sigma\eta)^{n}e^{(\sigma\eta)^{2}/2}}{\sigma\sqrt{2\pi}}I_{n-1}(x;-\eta,-\frac{1}{\sigma},-\sigma\eta)dt$$

by (B6). We can compute
$P(Z-\sum_{i=1}^{n}\xi_{i}\geq x)$ similarly.

Theorem B.1. With $\pi_{n}:= P(N(t)=n)=e^{-\lambda T}(\lambda T)^{n}/n!$ and $I_{n}$ in Proposition B.
, we have

$$P(Z(T)\geq a)=\frac{e^{(\sigma \eta_{1})^{2} T/2}}{\sigma \sqrt{2 \pi T}} \sum_{n=1}^{\infty} \pi_{n} \sum_{k=1}^{n} P_{n,k}(\sigma\sqrt{T}\eta_{1})^{k}\times I_{k-1}(a-\mu T; -\eta_{1},-\frac{1}{\sigma\sqrt{T}},-\sigma\eta_{1}\sqrt{T})$$

$$+\frac{e^{(\sigma\eta_{2})^{2}T/2}}{\sigma\sqrt{2\pi T}}\sum_{n=1}^{\infty}\pi_{n}\sum_{k=1}^{n}Q_{n,k}(\sigma\sqrt{T}\eta_{2})^{k}$$

$$\times I_{k-1}(a-\mu T; \eta_{2},\frac{1}{\sigma\sqrt{T}},-\sigma\eta_{2}\sqrt{T})$$

$$+\pi_{0}\phi(-\frac{a-\mu T}{\sigma\sqrt{T}})$$

Proof by the decomposition (B2)



%\end{document}
 %
%\ifthenelse{\value{noa} = 1}
%%...................then
%{\chapter*{APPENDIX: THIS IS THE FIRST APPENDIX}
%\addcontentsline{toc}{chapter}{APPENDIX: THIS IS THE FIRST APPENDIX}
%\chaptermark{Appendix}
%\markboth{Appendix}{Appendix}
%\setcounter{chapter}{1}}
%%...................else
{\chapter{THIS IS THE FIRST APPENDIX}}

%%%%%%%%%%%%%%%%%%%%%%%%%%%%%%%%%%%%%%%

%ADD LABEL

%%%%%%%%%%%%%%%%%%%%%%%%%%%%%%%%%%%%%%%

Proof of Proposition 1.

(1) Since B(T,T)=1, Equation (8) yields

$$B(t,T)=e^{\theta (T-t)}\frac{E((\delta(T))^{\alpha-1}|\mathfrak{F}_{t})}{(\delta(t))^{\alpha-1}}$$

Using the fact that

$$(\frac{\delta(T)}{\delta(t)})^{\alpha-1}=exp{(\alpha-1)(\mu_{1}-\frac{1}{2}\sigma_{1}^{2})(T-t)
+\sigma_{1}(\alpha-1)(W_{1}(T)-W_{1}(t))}\prod_{i=N(t)+1}^{N(T)}\widetilde{V}_{i}^{\alpha-1}$$

$$E(\prod_{i=N(t)+1}^{N(t)}\widetilde{V}_{i}^{\alpha-1})=\sum_{j=0}{\infty}e^{-\lambda(T-t)}\frac{[\lambda(T-t)]^{j}}{j!}{\zeta_{1}^{(\alpha-1)}+1}^{j}$$
$$=exp{\lambda\zeta_{1}^{(\alpha-1)}(T-t)}$$

First equation yields

$$B(t,T)=exp[-(T-t){\theta -(\alpha-1)(\mu_{1}-\frac{1}{2}\sigma_{1}^{2})-\frac{1}{2}\sigma_{1}^{2}(\alpha-1)^{2}-\lambda\zeta_{1}^{(\alpha-1)}}]$$

Note that it implies

$$e^{-r(T-t)}=E(U_{c}(\delta(T),T)/U_{c}(\delta(t),t)|\mathfrak{F}_{t})$$

which shows that Z(t) is a martingale under P. Furthermore, it leads to

$$Z(t)=(\delta(0))^{\alpha-1}e^{(r-\theta)t}exp{(\alpha-1)(\mu_{1}-\frac{1}{2}\sigma_{1}^{2})t +\sigma_{1}(\alpha-1)(W_{1}(t))}\prod_{i=1}^{N(t)}\widetilde{V}_{i}^{\alpha-1}$$
$$=(\delta(0))^{\alpha-1}exp{-\frac{1}{2}\sigma_{1}^{2}(\alpha-1)^{2}-\lambda\zeta_{1}^{(\alpha-1)}}t
+\sigma_{1}(\alpha-1)(W_{1}(t))\prod_{i=1}^{N(t)}\widetilde{V}_{i}^{\alpha-1}$$

Now

$$\psi_{s}(t)=\frac{E(U_{c}(\delta(T),T)\psi_{s}(T)|\mathfrak{F}_{t})}{(U_{c}(\delta(t),t))}=e^{-rT}E\{\frac{Z(T)}{Z(t)}\psi_{s}(T)|\mathfrak{F}_{t}\}$$
$$=e^{-rT}E^{*}(\psi_{s}(T)|\mathfrak{F}_{t})$$

Proof of Theorem 1. The Girsanov theorem for jump-diffusion processes tells us that under $P^{*}$, $W_{1}^{\prime}(t)= W_{1}(t)-\sigma_{1}(\alpha-1)t$ is a new Brownian motion and under $P^{*}$ the jump rate of N(t) is $\lambda^{*}=\lambda E(\widetilde{V}_{i}^{\alpha-1})=\lambda (\zeta_{1}^{(\alpha-1)}+1)$ and
$\widetilde{V}_{i}$ has a new density $f_{\widetilde{V}}^{*}(x)=(1/(\zeta_{1}^{(\alpha-1)}+1))x^{\alpha-1}f_{\widetilde{V}}(x)$. Therefore, dynamics of S(t) is given by

$$\frac{dS(t)}{S(t-)}=\mu dt+\sigma\{ \rho dW_{1}(t)+\sqrt{1-\rho^{2}} dW_{2}(t)\} + \Delta (\sum_{i=1}^{N(t)}(V_{i}^{\beta}-1))$$
$$=\{\mu + \sigma_{1}\sigma\rho (\alpha-1)dt + \sigma\{\rho dW_{1}^{\prime}(t)+\sqrt{1-\rho^{2}} dW_{2}(t)\}+ \Delta (\sum_{i=1}^{N(t)}(V_{i}^{\beta}-1))$$

Because

$$E^{*}(\widetilde{V}_{i}^{\beta})=\int_{0}^{\infty}x^{\beta}\frac{1}{\zeta_{1}^{(\alpha-1)}+1}x^{(\alpha-1)}f_{\widetilde{V}}(x)dx$$
$$=\frac{1}{\zeta_{1}^{(\alpha-1)}+1}E(\widetilde{V}^{\alpha+\beta-1})=\frac{\zeta_{1}^{\alpha+\beta-1}+1}{\zeta_{1}^{\alpha-1}+1}$$

we have $$\lambda^{*}\{E^{*}(\widetilde{V}^{\beta})-1\}=\lambda(\zeta_{1}^{\alpha+\beta-1}-\zeta_{1}^{\alpha-1})$$.

Therefore

$$\frac{dS(t)}{S(t-)}=\{\mu + \sigma_{1}\sigma\rho (\alpha-1)+ \lambda(\zeta_{\alpha+\beta-1}-\zeta_{\alpha-1})\}dt$$
$$-\lambda^{*}\{E^{*}(\widetilde{V}^{\beta})-1\}dt+\sigma\{\rho dW_{1}^{\prime}(t)+\sqrt{1-\rho^{2}} dW_{2}(t)\}+ \Delta (\sum_{i=1}^{N(t)}(V_{i}^{\beta}-1))$$

Hence to satisfy the rational equilibrium requirement $S(t)=e^{-r(T-t)}E^{*}(S(T)|\mathfrak{F})$ we must have $\mu + \sigma_{1}\sigma\rho (\alpha-1)+ \lambda(\zeta_{\alpha+\beta-1}-\zeta_{\alpha-1})=r$

So, under the measure $P^{*}$, the dynamics of S(t) is given by

$$\frac{dS(t)}{S(t-)}=rdt-\lambda^{*}\{E^{*}(\widetilde{V}^{\beta})-1\}dt+\sigma\{\rho dW_{1}^{\prime}(t)+\sqrt{1-\rho^{2}} dW_{2}(t)\}+ \Delta (\sum_{i=1}^{N(t)}(V_{i}^{\beta}-1))$$  %
\chapter{DERIVATION OF THE $\Upsilon$ FUNCTION}%
\label{appendixB}

%\clearpage %remove this command if your appendix doesn't start with a landscaped page!!!!!
%\thispagestyle{plain}
%\begin{landscape}
%\begin{figure}

% \begin{center}
  %  \includegraphics[width=6in]{LaTeX2e_logo.eps}
   % \caption{\LaTeX 2\ensuremath{\epsilon.} logo}\label{biglogo}
  %\end{center}
%\end{figure}
%\end{landscape}

%%%%%%%%%%%%%%%%%%%%%%%%%%%%%%%%%%%%%%%%%%%%%%%%%%%%%%%%%%%%%%%%%%%%%%%%%%%%%%%%%%%%%%%%%%%%%%%%%%

%ADD LABEL

%%%%%%%%%%%%%%%%%%%%%%%%%%%%%%%%%%%%%%%%%%%%%%%%%%%%%%%%%%%%%%%%%%%%%%%%%%%%%%%%%%%%%%%%%%%%%%%%%%

We first decompose the sum of the double exponential random variables.

The memoryless property of exponential random variables yields $(\xi^{+}-\xi^{-}|\xi^{+}>\xi^{-})=^{d}\xi^{+}$ and $(\xi^{+}-\xi^{-}|\xi^{+}<\xi^{-})=^{d}-\xi^{-}$, thus leading to the conclusion that

\begin{equation*}
\xi^{+}-\xi^{-} =\left\{
\begin{array}{rl}
\xi^{+} & \text{with probability $\eta_{2}/(\eta_{1}+\eta_{2})$ }\\
-\xi^{-} & \text{with probability $\eta_{1}/(\eta_{1}+\eta_{2})$ }
\end{array}\right\}.
\end{equation*}

because the probabilities of the events $\xi^{+}>\xi^{-}$ and $\xi^{+}<\xi^{-}$ are $\eta_{2}/(\eta_{1}+\eta_{2})$ and $\eta_{1}/(\eta_{1}+\eta_{2})$, respectively. The following proposition extends (B.1.)

Proposition B.1. For every $n\geq1$, we have the following decomposition

\begin{equation*}
\sum_{i=1}^{n}Y_{i}=^{d}\left\{
\begin{array}{rl}
\sum_{i=1}^{k}\xi_{i}^{+} & \text{with probability $P_{n,k},k=1,2,...,n$ }\\
-\sum_{i=1}^{k}\xi_{i}^{-} & \text{with probability $Q_{n,k},k=1,2,...,n$ }
\end{array}\right\}.
\end{equation*}

where $P_{n,k}$ and $Q_{n,k}$ are given by

$$P_{n,k}=\sum_{i=k}^{n-1}\binom {n-k-1} {i-k}\binom {n} {i}(\frac{\eta_{1}}{\eta_{1}+\eta_{2}})^{i-k}(\frac{\eta_{2}}{\eta_{1}+\eta_{2}})^{n-i}p^{i}q^{n-i}$$

$$1\leq k\leq n-1$$

$$Q_{n,k}=\sum_{i=k}^{n-1}\binom {n-k-1} {i-k}\binom {n} {i}(\frac{\eta_{1}}{\eta_{1}+\eta_{2}})^{n-i}(\frac{\eta_{2}}{\eta_{1}+\eta_{2}})^{i-k}p^{n-i}q^{i}$$

$$1\leq k\leq n-1, P_{n,n}=p^{n},Q_{n,n}=q^{n}$$

and $\binom{0}{0}$ is defined to be one. Hence $\xi_{i}^{+}$ and $\xi_{i}^{-}$ are i.i.d. exponential random variables with rates $\eta_{1}$ and $\eta_{2}$, respectively.

As a key step in deriving closed-form solutions for call and put options, this proposition indicates that the sum of the i.i.d. double exponential random variable can be written, in distribution, as a randomly mixed gamma random variable. To prove Proposition B.1, the following lemma is needed.

Lemma B.1.

$$\sum_{i=1}^{n}\xi_{i}^{+}-\sum_{i=1}^{n}\xi_{i}^{-}$$

\begin{equation*}
=^{d}\left\{
\begin{array}{rl}
\sum_{i=1}^{k}\xi_{i} & \text{with probability $\binom {n-k+m-1} {m-1}(\frac{\eta_{1}}{\eta_{1}+\eta_{2}})^{n-k}(\frac{\eta_{2}}{\eta_{1}+\eta_{2}})^{m}, k=1,...,n$ }\\
-\sum_{i=1}^{l}\xi_{i} & \text{with probability $\binom {n-l+m-1} {n-1}(\frac{\eta_{1}}{\eta_{1}+\eta_{2}})^{n}(\frac{\eta_{2}}{\eta_{1}+\eta_{2}})^{m-l}, l=1,...,m$ }
\end{array}\right\}.
\end{equation*}

We prove it by introducing the random variables $A(n,m) = \sum_{i=1}^{n}\xi_{i}-sum_{j=1}^{m}\tilde{\xi}_{j}$ Then

\begin{equation*}
A(n,m) =^{d}\left\{
\begin{array}{rl}
A(n-1,m-1)+\xi^{+} & \text{with probability $\eta_{2}/(\eta_{1}+\eta_{2})$ }\\
A(n-1,m-1)-\xi^{-} & \text{with probability $\eta_{1}/(\eta_{1}+\eta_{2})$ }
\end{array}\right\}.
\end{equation*}

\begin{equation*}
 =^{d}\left\{
\begin{array}{rl}
A(n,m-1) & \text{with probability $\eta_{2}/(\eta_{1}+\eta_{2})$ }\\
A(n-1,m) & \text{with probability $\eta_{1}/(\eta_{1}+\eta_{2})$ }
\end{array}\right\}.
\end{equation*}

via B.1.. Now suppose horizontal axis that are representing the number of $\{\zeta_{i}^{+}\}$ and vertical axis representing the number of $\{\zeta_{i}^{-}\}$. Suppose we have a random walk on the integer lattice points. Starting from any point $(n,m),n,m \geq 1$, the random walk goes either one step to the left with probability $\eta_{1}/(\eta_{1}+\eta_{2})$ or one step down with probability $\eta_{2}/(\eta_{1}+\eta_{2})$, and the random walks stops once it reaches the horizontal or vertical axis. For any path from (n,m) to (k,0) , $1 \geq k \geq n$, it must reach (k,1) first before it makes a final move to (k,0). Furthermore, all the paths going from (n,m) to (k,1) must have exactly n-k lefts and m-1 downs, whence the total number of such paths is $\binom {n-k+m-1}{m-1}$. Similarly the total number of paths from (n,m) to (0,l) , $1 \geq l \geq m$, is $\binom {n-l+m-1}{n-1}$. Thus

\begin{equation*}
A(n,m)=^{d}\left\{
\begin{array}{rl}
\sum_{i=1}^{k}\xi_{i} & \text{with probability $\binom {n-k+m-1} {m-1}(\frac{\eta_{1}}{\eta_{1}+\eta_{2}})^{n-k}(\frac{\eta_{2}}{\eta_{1}+\eta_{2}})^{m}, k=1,...,n$ }\\
-\sum_{i=1}^{l}\xi_{i} & \text{with probability $\binom {n-l+m-1} {n-1}(\frac{\eta_{1}}{\eta_{1}+\eta_{2}})^{n}(\frac{\eta_{2}}{\eta_{1}+\eta_{2}})^{m-l}, l=1,...,m$ }
\end{array}\right\}.
\end{equation*}

and the lemma is proven.

Now, let's prove the proposition B.1. By the same analogy used in Lemma B.1 to compute probability $P_{n,m},1\geq k \geq n$, the probability weight assigned to $\sum_{i=1}^{k}\xi_{i}^{+}$ when we decompose $\sum_{i=1}^{k}Y_{i}$, it is equivalent to consider the probability of the random walk ever reach (k,0) starting from the point (i,n-i) being $\binom {n}{i}p^{i}q^{n-i}$. Note that the point (k,0) can only be reached from point (i,n-i) such that $k \geq i \geq n-1$, because the random walk can only go left or down, and stops once it reaches the horizontal axis. Therefore, for $1 \geq k \geq n-1$, (B3) leads to

$$P_{n,k}=\sum_{i=k}{n-1}P(going from (i,n-i) to (k,0)). P(starting from (i,n-i))$$

$$=\sum_{i=k}^{n-1}\binom {i+(n-i)-k-1} {(n-i)-1}\binom {n} {i}(\frac{\eta_{1}}{\eta_{1}+\eta_{2}})^{i-k}(\frac{\eta_{2}}{\eta_{1}+\eta_{2}})^{n-i}p^{i}q^{n-i}$$

$$=\sum_{i=k}^{n-1}\binom {n-k-1} {n-i-1}\binom {n} {i}(\frac{\eta_{1}}{\eta_{1}+\eta_{2}})^{i-k}(\frac{\eta_{2}}{\eta_{1}+\eta_{2}})^{n-i}p^{i}q^{n-i}$$

$$=\sum_{i=k}^{n-1}\binom {n-k-1} {i-k}\binom {n} {i}(\frac{\eta_{1}}{\eta_{1}+\eta_{2}})^{i-k}(\frac{\eta_{2}}{\eta_{1}+\eta_{2}})^{n-i}p^{i}q^{n-i}$$

Of course $P_{n,n}=p^{n}$. Similarly, we can compute $Q_{n,k}$:

$$Q_{n,k}=\sum_{i=k}{n-1}P(going from (n-i,i) to (0,k)). P(starting from (n-i,i))$$

$$=\sum_{i=k}^{n-1}\binom {i+(n-i)-k-1} {(n-i)-1}\binom {n} {n-i}(\frac{\eta_{1}}{\eta_{1}+\eta_{2}})^{n-i}(\frac{\eta_{2}}{\eta_{1}+\eta_{2}})^{i-k}p^{n-i}q^{i}$$

$$=\sum_{i=k}^{n-1}\binom {n-k-1} {i-k}\binom {n} {i}(\frac{\eta_{1}}{\eta_{1}+\eta_{2}})^{n-i}(\frac{\eta_{2}}{\eta_{1}+\eta_{2}})^{i-k}p^{n-i}q^{i}$$

with $Q_{n,n}=q^{n}$. Incidentally, we have also got $\sum{k=1}{n}(P_{n,k}+Q_{n,k})=1$

B.2. Let's develop now the results on Hh functions.
First of all, note that $Hh_{n}(x)\rightarrow 0$, as $x \rightarrow \infty$, for $n \geq -1$; and $Hh_{n}(x) \rightarrow \infty$, as $x \rightarrow -\infty$, for $n \geq -1$; and $Hh_{0}(x)=\sqrt{2\pi} \phi(-x) \rightarrow \sqrt{2\pi}$, as $x \rightarrow -\infty$. Also, for every $n \geq -1$, as $x \rightarrow \infty$,

$$lim Hh_{n}(x)/\{\frac{1}{x^{n+1}}e^{-\frac{x^{2}}{2}}\}=1$$

and as $x \rightarrow \infty$

$$Hh_{n}(x)=O(|x|^{n})$$

Here (B4) is clearly true for $n=-1$, while for $n \geq 0$ note that as $x\rightarrow _\infty$,

$$Hh_{n}(x)=\frac{1}{n!}\int_{x}{\infty}(t-x)^{n}e^{-\frac{t^{2}}{2}}dt$$

$$\leq \frac{2^{n}}{n!}\int_{-\infty}^{\infty}|t|^{n}e^{-t^{2}}{2}dt+\frac{2^{n}}{n!}\int{-\infty}{\infty}|x|^{n}e^{-t^{2}}{2}dt=O(|x|^{n})$$

For option pricing it is important to evaluate the integral $I_{n}(c;\alpha;\beta;\delta)$,

$$I_{n}(c;\alpha;\beta;\delta)=\int_{c}{\infty}e^{\alpha x}Hh_{n}(\beta x-\delta)dx, n\geq 0$$

for arbitrary constants $\alpha, c$ and $\beta$.
 %
%\chapter{DERIVATION OF THE $\Upsilon$ FUNCTION}%
\label{appendixC}

%\clearpage %remove this command if your appendix doesn't start with a landscaped page!!!!!
%\thispagestyle{plain}
%\begin{landscape}
%\begin{figure}

 % \begin{center}
  %  \includegraphics[width=6in]{LaTeX2e_logo.eps}
   % \caption{\LaTeX 2\ensuremath{\epsilon.} logo}\label{biglogo}
  %\end{center}
%\end{figure}
%\end{landscape}

%%%%%%%%%%%%%%%%%%%%%%%%%%%%%%%%%%%%%%%%%%%%%%%%%%%%%%%%%%%%%%%%%%%%%%%%%%%%%%%%%%%%%%%%%%%%%%%%%%


%ADD LABEL

%%%%%%%%%%%%%%%%%%%%%%%%%%%%%%%%%%%%%%%%%%%%%%%%%%%%%%%%%%%%%%%%%%%%%%%%%%%%%%%%%%%%%%%%%%%%%%%%%%

\proposition{The Upsilon Function}\label{first}

(1) If $\beta>0$ and $\alpha\neq0$, then for all $n\geq-1$,

$$I_{n}(c;\alpha; \beta; \delta) = - \frac{e^{\alpha c}}{\alpha} \sum_{i=0}^{n}(\frac{\beta}{\alpha})^{n-i} Hh_{i}(\beta c -\delta)$$
$$+ (\frac{\beta}{\alpha})^{n+1} \frac{\sqrt{2 \pi}}{\beta} e^{\frac{\alpha \delta}{\beta}+\frac{\alpha^{2}}{2\beta^{2}}} \phi(-\beta c + \delta + \frac{\alpha}{\beta})$$

(2) If $\beta<0$ and $\alpha<0$, then for all $x \geq -1$

$$I_{n}(c;\alpha; \beta; \delta) = - \frac{e^{\alpha c}}{\alpha} \sum_{i=0}^{n}(\frac{\beta}{\alpha})^{n-i} Hh_{i}(\beta c -\delta)$$
$$- (\frac{\beta}{\alpha})^{n+1} \frac{\sqrt{2 \pi}}{\beta} e^{\frac{\alpha \delta}{\beta}+\frac{\alpha^{2}}{2\beta^{2}}} \phi(\beta c - \delta - \frac{\alpha}{\beta})$$

\begin{proof}{Case 1.}
$\beta>0$ and $\alpha\neq0$. Since, for any constant $\alpha$ and $n \geq 0$, $e^{\alpha x} Hh_{n}(\beta x - \delta) \rightarrow 0$ as $x \rightarrow \infty$ thanks to (B4), integration by parts leads to

$$I_{n}=-\frac{1}{\alpha}Hh(\beta c -\delta) e^{\alpha c} + \frac{\beta}{\alpha}\int_{c}^{\infty} e^{\alpha x} Hh_{n-1}(\beta c - \delta)dx$$

In other words, we have a recursion, for $n \geq 0$, $I_{n}=-(e^{\alpha c}{\alpha})Hh_{n}(\beta c - \delta) + (\frac{\beta}{\alpha})I_{n-1}$ with

$$I_{-1}=\sqrt{2 \pi} \int_{c}{\infty}e^{\alpha x}\varphi(-\beta x +\delta)dx$$

$$=\frac{\sqrt{2 \pi}}{\beta} e^{\frac{\alpha \delta}{\beta}+\frac{\alpha^{2}}{2 \beta^{2}}}\phi(-\beta c + \delta +\frac{\alpha}{\beta})$$

Solving it yields, for $n \geq -1$,

$$I_{n}=-\frac{e^{\alpha c}}{\alpha}\sum_{i=0}^{n}(\frac{\beta}{\alpha})^{i}Hh_{n-i}(\beta c+\delta) + (\frac{\beta}{\alpha})^{n+1}I_{-1}$$

$$=-\frac{e^{\alpha c}}{\alpha}\sum_{i=0}^{n}(\frac{\beta}{\alpha})^{n-i} Hh_{i}(\beta c+\delta)$$
$$+ (\frac{\beta}{\alpha})^{n+1}\frac{\sqrt{2 \pi}}{\beta} e^{\frac{\alpha \delta}{\beta}+\frac{\alpha^{2}}{2 \beta^{2}}}\phi(-\beta c + \delta +\frac{\alpha}{\beta})$$

where the sum over an empty set is defined to be zero.
\end{proof}

Case2. $\beta<0$ and $\alpha<0$. In this case, we must also have, for $n \geq 0$ and any constant $\alpha<0, e^{\alpha x}Hh_{n}(\beta x -\delta) \rightarrow 0$ as
$x \rightarrow \infty$, thanks to (B5). Using integration by parts, we again have the same recursion, for $n \geq 0, I_{n}=-(e^{\alpha c}/\alpha)Hh_{n}(\beta c - \delta)+(\beta / \alpha)I_{n-1}$, but with a different initial condition

$$I_{-1}=\sqrt{2 \pi}\int_{c}^{\infty}e^{\alpha x}\varphi(-\beta x + \delta)dx$$

$$=-\frac{\sqrt{2 \pi}}{\beta} exp\{\frac{\alpha \delta}{\beta}+\frac{\alpha^{2}}{2 \beta^{2}}\}\phi(\beta c - \delta -\frac{\alpha}{\beta})$$

Solving it yields (B8), for $n \geq -1$.

Finally, we sum the double exponential and the normal random variables

Proposition B.3.

Suppose $\{\xi_{1},\xi_{2},...\}$ is a sequence of i.i.d. exponential random variables with rate $\eta>0$, and Z is a normal variable with distribution $N(0,\sigma^{2})$. Then for every $ n \geq 1$, we have: (1) The density functions are given by:

$$f_{Z+\sum_{i=1}^{n}\xi_{i}}(t)=(\sigma\eta)^{n}\frac{e^{(\sigma\eta)^{2}/2}}{\sigma\sqrt{2\pi}}e^{-t\eta}Hh_{n-1}(-\frac{t}{\sigma}+\sigma\eta)$$

$$f_{Z-\sum_{i=1}^{n}\xi_{i}}(t)=(\sigma\eta)^{n}\frac{e^{(\sigma\eta)^{2}/2}}{\sigma\sqrt{2\pi}}e^{-t\eta}Hh_{n-1}(\frac{t}{\sigma}+\sigma\eta)$$

(2) The tail probabilities are given by

$$P(Z+\sum_{i=1}^{n}\xi_{i}\geq x) = (\sigma\eta)^{n}\frac{e^{(\sigma\eta)^{2}/2}}{\sigma\sqrt{2\pi}}e^{-t\eta}I_{n-1}(x;-\eta,-\frac{1}{\sigma},-\sigma\eta)$$

$$P(Z-\sum_{i=1}^{n}\xi_{i}\geq x) = (\sigma\eta)^{n}\frac{e^{(\sigma\eta)^{2}/2}}{\sigma\sqrt{2\pi}}e^{-t\eta}I_{n-1}(x;\eta,\frac{1}{\sigma},-\sigma\eta)$$

Proof. Case 1. The densities of $Z+\sum_{i=1}^{n}\xi_{i}$, and $Z-\sum_{i=1}^{n}\xi_{i}$. We have

$$f_{Z+\sum_{i=1}^{n}\xi_{i}}(t)=\int_{-\infty}^{\infty}f_{\sum_{i=1}^{n}\xi_{i}}(t-x)f_{Z}(x)dx$$

$$=e^{-t\eta}(\eta^{n})\int_{-\infty}{t}\frac{e^{x\eta}(t-x)^{n-1}}{(n-1)!}\frac{1}{\sigma\sqrt{2\pi}}e^{-x^{2}/(2\sigma^{2})}dx$$

$$=e^{-t\eta}(\eta^{n})e^{(\sigma\eta)^{2}/(2)}\int_{-\infty}{t}\frac{(t-x)^{n-1}}{(n-1)!}\frac{1}{\sigma\sqrt{2\pi}}e^{-(x-\sigma^{2}\eta)^{2}/(2\sigma^{2})}dx$$

Letting $y=(x-\sigma^{2}\eta)/\sigma$ yields

$$f_{Z+\sum_{i=1}^{n}\xi_{i}}(t)=e^{-t\eta}(\eta^{n})e^{(\sigma\eta)^{2}/(2)}\sigma^{n-1}$$

$$\times\int_{-\infty}^{t/\sigma-\sigma\eta}\frac{(t/\sigma - y -\sigma\eta)^{n-1}}{(n-1)!}\frac{1}{\sqrt{2\pi}}e^{-y^{2}/2}dy$$

$$=\frac{e^{(\sigma\eta)^{2}/2}}{\sqrt{2\pi}}(\sigma^{n-1}\eta^{n})e^{-t\eta}Hh_{n-1}(-t/\sigma + \sigma\eta)$$

because $(1/(n-1)!)\int_{-\infty}{a}(a-y)^{n-1}e^{-y^{2}/2}dy=Hh_{n-1}(a)$. The derivation of $f_{Z+\sum_{i=1}^{n}\xi_{i}}(t)$ is similar.

Case 2. $P(Z+\sum_{i=1}^{n}\xi_{i}\geq x)$ and $P(Z-\sum_{i=1}^{n}\xi_{i}\geq x)$. From (B9), it is clear that

$$P(Z+\sum_{i=1}^{n}\xi_{i}\geq x)=\frac{(\sigma\eta)^{n}e^{(\sigma\eta)^{2}/2}}{\sigma\sqrt{2\pi}}\int_{x}^{\infty}e^{(-i\eta)}Hh_{n-1}(-\frac{t}{\sigma}+\sigma\eta)dt$$

$$=\frac{(\sigma\eta)^{n}e^{(\sigma\eta)^{2}/2}}{\sigma\sqrt{2\pi}}I_{n-1}(x;-\eta,-\frac{1}{\sigma},-\sigma\eta)dt$$

by (B6). We can compute
$P(Z-\sum_{i=1}^{n}\xi_{i}\geq x)$ similarly.

\theorem{Theorem} With $\pi_{n}:= P(N(t)=n)=e^{-\lambda T}(\lambda T)^{n}/n!$ and $I_{n}$ in Proposition \ref{first}.
, we have

$$P(Z(T)\geq a)=\frac{e^{(\sigma \eta_{1})^{2} T/2}}{\sigma \sqrt{2 \pi T}} \sum_{n=1}^{\infty} \pi_{n} \sum_{k=1}^{n} P_{n,k}(\sigma\sqrt{T}\eta_{1})^{k}\times I_{k-1}(a-\mu T; -\eta_{1},-\frac{1}{\sigma\sqrt{T}},-\sigma\eta_{1}\sqrt{T})$$

$$+\frac{e^{(\sigma\eta_{2})^{2}T/2}}{\sigma\sqrt{2\pi T}}\sum_{n=1}^{\infty}\pi_{n}\sum_{k=1}^{n}Q_{n,k}(\sigma\sqrt{T}\eta_{2})^{k}$$

$$\times I_{k-1}(a-\mu T; \eta_{2},\frac{1}{\sigma\sqrt{T}},-\sigma\eta_{2}\sqrt{T})$$

$$+\pi_{0}\phi(-\frac{a-\mu T}{\sigma\sqrt{T}})$$

Proof by the decomposition (B2)

$$P(Z(T) \geq a)= \sum_{n=0}^{\infty}\pi_{n} P(\mu T +\sigma\sqrt{T} Z + \sum_{j=1}^{n}Y_{j} \geq a)$$

$$=\pi_{0}P(\mu T +\sigma\sqrt{T} Z  \geq a)$$

$$+\sum_{n=1}^{\infty}\pi_{n}\sum_{k=1}^{n}P_{n,k} P(\mu T +\sigma\sqrt{T} Z + \sum_{j=1}^{n}\xi_{j}^{+} \geq a)$$

$$+\sum_{n=1}^{\infty}\pi_{n}\sum_{k=1}^{n}Q_{n,k} P(\mu T +\sigma\sqrt{T} Z - \sum_{j=1}^{n}\xi_{j}^{-} \geq a)$$

The result now follows via (B11) and (B12) for $\eta_{1} > 1$ and $\eta_{2} >0$.


  %These files aren't included in the template
%%\documentclass{article}
%\usepackage[all]{xy}
%\usepackage{amssymb}
%\usepackage{amsmath}
%\usepackage{amsfonts}
%\usepackage{amsthm}
%\usepackage{amscd}
%\usepackage{eucal}
%\usepackage[dvips]{epsfig}
%\usepackage{graphicx}
%\usepackage{ulem}
%\usepackage{wrapfig}
%\addtolength{\hoffset}{-2cm}
%\addtolength{\topmargin}{-2.8cm}
%\addtolength{\textwidth}{3 cm}
%\addtolength{\textheight}{6.2 cm}
%
%\def\ii{{\bf i}}
%\def\jj{{\bf j}}
%\def\kk{{\bf k}}
%\def\aa{{\bf a}}
%\def\bb{{\bf b}}
%\def\nn{{\bf n}}
%\def\uu{{\bf u}}
%\def\vv{{\bf v}}
%\def\rr{{\bf r}}
%\def\ff{{\bf F}}
%
%\begin{document}

%%%%%%%%%%%%%%%%%%%%%%%%%%%%%%%%%%%%%%%%%%%%%%%

\chapter{DERIVATION OF THE $\Upsilon$ FUNCTION}%
\label{appendixB}

%%\clearpage %remove this command if your appendix doesn't start with a landscaped page!!!!!
%%\thispagestyle{plain}
%%\begin{landscape}
%%\begin{figure}

 %% \begin{center}
  %%  \includegraphics[width=6in]{LaTeX2e_logo.eps}
   %% \caption{\LaTeX 2\ensuremath{\epsilon.} logo}\label{biglogo}
  %%\end{center}
%%\end{figure}
%%\end{landscape}

%%%%%%%%%%%%%%%%%%%%%%%%%%%%%%%%%%%%%%%%%%%%%%%%%%%%%%%%%%%%%%%%%%%%%%%%%%%%%%%%%%%%%%%%%%%%%%%%%%


%ADD LABEL


Proposition B.3.

Suppose $\{\xi_{1},\xi_{2},...\}$ is a sequence of i.i.d. exponential random variables with rate $\eta>0$, and Z is a normal variable with distribution $N(0,\sigma^{2})$. Then for every $ n \geq 1$, we have: (1) The density functions are given by:

$$f_{Z+\sum_{i=1}^{n}\xi_{i}}(t)=(\sigma\eta)^{n}\frac{e^{(\sigma\eta)^{2}/2}}{\sigma\sqrt{2\pi}}e^{-t\eta}Hh_{n-1}(-\frac{t}{\sigma}+\sigma\eta)$$

$$f_{Z-\sum_{i=1}^{n}\xi_{i}}(t)=(\sigma\eta)^{n}\frac{e^{(\sigma\eta)^{2}/2}}{\sigma\sqrt{2\pi}}e^{-t\eta}Hh_{n-1}(\frac{t}{\sigma}+\sigma\eta)$$

(2) The tail probabilities are given by

$$P(Z+\sum_{i=1}^{n}\xi_{i}\geq x) = (\sigma\eta)^{n}\frac{e^{(\sigma\eta)^{2}/2}}{\sigma\sqrt{2\pi}}e^{-t\eta}I_{n-1}(x;-\eta,-\frac{1}{\sigma},-\sigma\eta)$$

$$P(Z-\sum_{i=1}^{n}\xi_{i}\geq x) = (\sigma\eta)^{n}\frac{e^{(\sigma\eta)^{2}/2}}{\sigma\sqrt{2\pi}}e^{-t\eta}I_{n-1}(x;\eta,\frac{1}{\sigma},-\sigma\eta)$$

Proof. Case 1. The densities of $Z+\sum_{i=1}^{n}\xi_{i}$, and $Z-\sum_{i=1}^{n}\xi_{i}$. We have

$$f_{Z+\sum_{i=1}^{n}\xi_{i}}(t)=\int_{-\infty}^{\infty}f_{\sum_{i=1}^{n}\xi_{i}}(t-x)f_{Z}(x)dx$$

$$=e^{-t\eta}(\eta^{n})\int_{-\infty}{t}\frac{e^{x\eta}(t-x)^{n-1}}{(n-1)!}\frac{1}{\sigma\sqrt{2\pi}}e^{-x^{2}/(2\sigma^{2})}dx$$

$$=e^{-t\eta}(\eta^{n})e^{(\sigma\eta)^{2}/(2)}\int_{-\infty}{t}\frac{(t-x)^{n-1}}{(n-1)!}\frac{1}{\sigma\sqrt{2\pi}}e^{-(x-\sigma^{2}\eta)^{2}/(2\sigma^{2})}dx$$

Letting $y=(x-\sigma^{2}\eta)/\sigma$ yields

$$f_{Z+\sum_{i=1}^{n}\xi_{i}}(t)=e^{-t\eta}(\eta^{n})e^{(\sigma\eta)^{2}/(2)}\sigma^{n-1}$$

$$\times\int_{-\infty}^{t/\sigma-\sigma\eta}\frac{(t/\sigma - y -\sigma\eta)^{n-1}}{(n-1)!}\frac{1}{\sqrt{2\pi}}e^{-y^{2}/2}dy$$

$$=\frac{e^{(\sigma\eta)^{2}/2}}{\sqrt{2\pi}}(\sigma^{n-1}\eta^{n})e^{-t\eta}Hh_{n-1}(-t/\sigma + \sigma\eta)$$

because $(1/(n-1)!)\int_{-\infty}{a}(a-y)^{n-1}e^{-y^{2}/2}dy=Hh_{n-1}(a)$. The derivation of $f_{Z+\sum_{i=1}^{n}\xi_{i}}(t)$ is similar.

Case 2. $P(Z+\sum_{i=1}^{n}\xi_{i}\geq x)$ and $P(Z-\sum_{i=1}^{n}\xi_{i}\geq x)$. From (B9), it is clear that

$$P(Z+\sum_{i=1}^{n}\xi_{i}\geq x)=\frac{(\sigma\eta)^{n}e^{(\sigma\eta)^{2}/2}}{\sigma\sqrt{2\pi}}\int_{x}^{\infty}e^{(-i\eta)}Hh_{n-1}(-\frac{t}{\sigma}+\sigma\eta)dt$$

$$=\frac{(\sigma\eta)^{n}e^{(\sigma\eta)^{2}/2}}{\sigma\sqrt{2\pi}}I_{n-1}(x;-\eta,-\frac{1}{\sigma},-\sigma\eta)dt$$

by (B6). We can compute
$P(Z-\sum_{i=1}^{n}\xi_{i}\geq x)$ similarly.

Theorem B.1. With $\pi_{n}:= P(N(t)=n)=e^{-\lambda T}(\lambda T)^{n}/n!$ and $I_{n}$ in Proposition B.
, we have

$$P(Z(T)\geq a)=\frac{e^{(\sigma \eta_{1})^{2} T/2}}{\sigma \sqrt{2 \pi T}} \sum_{n=1}^{\infty} \pi_{n} \sum_{k=1}^{n} P_{n,k}(\sigma\sqrt{T}\eta_{1})^{k}\times I_{k-1}(a-\mu T; -\eta_{1},-\frac{1}{\sigma\sqrt{T}},-\sigma\eta_{1}\sqrt{T})$$

$$+\frac{e^{(\sigma\eta_{2})^{2}T/2}}{\sigma\sqrt{2\pi T}}\sum_{n=1}^{\infty}\pi_{n}\sum_{k=1}^{n}Q_{n,k}(\sigma\sqrt{T}\eta_{2})^{k}$$

$$\times I_{k-1}(a-\mu T; \eta_{2},\frac{1}{\sigma\sqrt{T}},-\sigma\eta_{2}\sqrt{T})$$

$$+\pi_{0}\phi(-\frac{a-\mu T}{\sigma\sqrt{T}})$$

Proof by the decomposition (B2)



%\end{document}

%%\ifthenelse{\value{noa} = 1}
%%...................then
%{\chapter*{APPENDIX: THIS IS THE FIRST APPENDIX}
%\addcontentsline{toc}{chapter}{APPENDIX: THIS IS THE FIRST APPENDIX}
%\chaptermark{Appendix}
%\markboth{Appendix}{Appendix}
%\setcounter{chapter}{1}}
%%...................else
{\chapter{THIS IS THE FIRST APPENDIX}}

%%%%%%%%%%%%%%%%%%%%%%%%%%%%%%%%%%%%%%%

%ADD LABEL

%%%%%%%%%%%%%%%%%%%%%%%%%%%%%%%%%%%%%%%

Proof of Proposition 1.

(1) Since B(T,T)=1, Equation (8) yields

$$B(t,T)=e^{\theta (T-t)}\frac{E((\delta(T))^{\alpha-1}|\mathfrak{F}_{t})}{(\delta(t))^{\alpha-1}}$$

Using the fact that

$$(\frac{\delta(T)}{\delta(t)})^{\alpha-1}=exp{(\alpha-1)(\mu_{1}-\frac{1}{2}\sigma_{1}^{2})(T-t)
+\sigma_{1}(\alpha-1)(W_{1}(T)-W_{1}(t))}\prod_{i=N(t)+1}^{N(T)}\widetilde{V}_{i}^{\alpha-1}$$

$$E(\prod_{i=N(t)+1}^{N(t)}\widetilde{V}_{i}^{\alpha-1})=\sum_{j=0}{\infty}e^{-\lambda(T-t)}\frac{[\lambda(T-t)]^{j}}{j!}{\zeta_{1}^{(\alpha-1)}+1}^{j}$$
$$=exp{\lambda\zeta_{1}^{(\alpha-1)}(T-t)}$$

First equation yields

$$B(t,T)=exp[-(T-t){\theta -(\alpha-1)(\mu_{1}-\frac{1}{2}\sigma_{1}^{2})-\frac{1}{2}\sigma_{1}^{2}(\alpha-1)^{2}-\lambda\zeta_{1}^{(\alpha-1)}}]$$

Note that it implies

$$e^{-r(T-t)}=E(U_{c}(\delta(T),T)/U_{c}(\delta(t),t)|\mathfrak{F}_{t})$$

which shows that Z(t) is a martingale under P. Furthermore, it leads to

$$Z(t)=(\delta(0))^{\alpha-1}e^{(r-\theta)t}exp{(\alpha-1)(\mu_{1}-\frac{1}{2}\sigma_{1}^{2})t +\sigma_{1}(\alpha-1)(W_{1}(t))}\prod_{i=1}^{N(t)}\widetilde{V}_{i}^{\alpha-1}$$
$$=(\delta(0))^{\alpha-1}exp{-\frac{1}{2}\sigma_{1}^{2}(\alpha-1)^{2}-\lambda\zeta_{1}^{(\alpha-1)}}t
+\sigma_{1}(\alpha-1)(W_{1}(t))\prod_{i=1}^{N(t)}\widetilde{V}_{i}^{\alpha-1}$$

Now

$$\psi_{s}(t)=\frac{E(U_{c}(\delta(T),T)\psi_{s}(T)|\mathfrak{F}_{t})}{(U_{c}(\delta(t),t))}=e^{-rT}E\{\frac{Z(T)}{Z(t)}\psi_{s}(T)|\mathfrak{F}_{t}\}$$
$$=e^{-rT}E^{*}(\psi_{s}(T)|\mathfrak{F}_{t})$$

Proof of Theorem 1. The Girsanov theorem for jump-diffusion processes tells us that under $P^{*}$, $W_{1}^{\prime}(t)= W_{1}(t)-\sigma_{1}(\alpha-1)t$ is a new Brownian motion and under $P^{*}$ the jump rate of N(t) is $\lambda^{*}=\lambda E(\widetilde{V}_{i}^{\alpha-1})=\lambda (\zeta_{1}^{(\alpha-1)}+1)$ and
$\widetilde{V}_{i}$ has a new density $f_{\widetilde{V}}^{*}(x)=(1/(\zeta_{1}^{(\alpha-1)}+1))x^{\alpha-1}f_{\widetilde{V}}(x)$. Therefore, dynamics of S(t) is given by

$$\frac{dS(t)}{S(t-)}=\mu dt+\sigma\{ \rho dW_{1}(t)+\sqrt{1-\rho^{2}} dW_{2}(t)\} + \Delta (\sum_{i=1}^{N(t)}(V_{i}^{\beta}-1))$$
$$=\{\mu + \sigma_{1}\sigma\rho (\alpha-1)dt + \sigma\{\rho dW_{1}^{\prime}(t)+\sqrt{1-\rho^{2}} dW_{2}(t)\}+ \Delta (\sum_{i=1}^{N(t)}(V_{i}^{\beta}-1))$$

Because

$$E^{*}(\widetilde{V}_{i}^{\beta})=\int_{0}^{\infty}x^{\beta}\frac{1}{\zeta_{1}^{(\alpha-1)}+1}x^{(\alpha-1)}f_{\widetilde{V}}(x)dx$$
$$=\frac{1}{\zeta_{1}^{(\alpha-1)}+1}E(\widetilde{V}^{\alpha+\beta-1})=\frac{\zeta_{1}^{\alpha+\beta-1}+1}{\zeta_{1}^{\alpha-1}+1}$$

we have $$\lambda^{*}\{E^{*}(\widetilde{V}^{\beta})-1\}=\lambda(\zeta_{1}^{\alpha+\beta-1}-\zeta_{1}^{\alpha-1})$$.

Therefore

$$\frac{dS(t)}{S(t-)}=\{\mu + \sigma_{1}\sigma\rho (\alpha-1)+ \lambda(\zeta_{\alpha+\beta-1}-\zeta_{\alpha-1})\}dt$$
$$-\lambda^{*}\{E^{*}(\widetilde{V}^{\beta})-1\}dt+\sigma\{\rho dW_{1}^{\prime}(t)+\sqrt{1-\rho^{2}} dW_{2}(t)\}+ \Delta (\sum_{i=1}^{N(t)}(V_{i}^{\beta}-1))$$

Hence to satisfy the rational equilibrium requirement $S(t)=e^{-r(T-t)}E^{*}(S(T)|\mathfrak{F})$ we must have $\mu + \sigma_{1}\sigma\rho (\alpha-1)+ \lambda(\zeta_{\alpha+\beta-1}-\zeta_{\alpha-1})=r$

So, under the measure $P^{*}$, the dynamics of S(t) is given by

$$\frac{dS(t)}{S(t-)}=rdt-\lambda^{*}\{E^{*}(\widetilde{V}^{\beta})-1\}dt+\sigma\{\rho dW_{1}^{\prime}(t)+\sqrt{1-\rho^{2}} dW_{2}(t)\}+ \Delta (\sum_{i=1}^{N(t)}(V_{i}^{\beta}-1))$$ 
